\documentclass[12pt,letterpaper]{article}

% Essential packages
\usepackage[utf8]{inputenc}
\usepackage[T1]{fontenc}
\usepackage{lmodern}
\usepackage[margin=1in]{geometry}
\usepackage{setspace}
\onehalfspacing

% Math packages
\usepackage{amsmath,amssymb,amsthm,mathtools}
\usepackage{bm}  % Bold math symbols

% Graphics and tables
\usepackage{graphicx}
\usepackage{float}  % For [H] figure placement
\usepackage{booktabs}
\usepackage{longtable}
\usepackage{array}

% References and citations
\usepackage[numbers,sort&compress]{natbib}
\usepackage{hyperref}
\hypersetup{
    colorlinks=true,
    linkcolor=blue,
    citecolor=blue,
    urlcolor=blue
}

% Section formatting - using standard LaTeX formatting

% Custom commands
\newcommand{\authdhd}{\text{AuDHD}}
\newcommand{\R}{\mathbb{R}}
\newcommand{\E}{\mathbb{E}}
\newcommand{\Prob}{\mathbb{P}}
\newcommand{\norm}[1]{\left\|#1\right\|}
\newcommand{\abs}[1]{\left|#1\right|}

% Theorem environments
\newtheorem{theorem}{Theorem}[section]
\newtheorem{lemma}[theorem]{Lemma}
\newtheorem{proposition}[theorem]{Proposition}
\newtheorem{corollary}[theorem]{Corollary}
\theoremstyle{definition}
\newtheorem{definition}[theorem]{Definition}
\newtheorem{example}[theorem]{Example}
\theoremstyle{remark}
\newtheorem{remark}[theorem]{Remark}

% Title information
\title{\textbf{Continuous Enrichment Stratification of Neurotransmitter Genes Reveals Density Peaks in Shared ADHD-Autism Genetic Architecture}}

\author{
    [Author Name]$^{1,2}$\thanks{Corresponding author: email@institution.edu}\\
    \\
    \small $^1$Department of Genetics, University\\
    \small $^2$Institute for Computational Biology
}

\date{\today}

\begin{document}

\maketitle

\begin{abstract}
\noindent
\textbf{Background:} Co-occurrence of autism spectrum disorder (ASD) and attention-deficit/hyperactivity disorder (ADHD) affects 50-70\% of autistic individuals, yet the genetic architecture underlying this comorbidity remains poorly characterized.

\noindent
\textbf{Methods:} We analyzed gene-level association patterns for 35 neurotransmitter pathway genes using GWAS summary statistics from ADHD (38,691 cases; Demontis 2023) and autism (18,381 cases; Grove 2019). Shared genetic contribution was quantified as the geometric mean of MAGMA-derived gene association scores ($-\log_{10}$ p-values) across disorders. Genes exhibited continuous stratification with five recurrent density peaks identified through clustering. Validation employed gene-level correlation with 11 independent cross-disorder GWAS studies, label permutation testing (10,000 iterations), and partial correlation controlling for gene length.

\noindent
\textbf{Results:} Gene association scores showed continuous stratification with five recurrent density peaks: glutamatergic-extreme (4 genes, mean=1006), GABAergic (3 genes, mean=633), serotonergic (1 gene, score=215), dopaminergic (4 genes, mean=197), and polygenic-background (23 genes, mean=84). Discovery scores strongly correlated with independent cross-disorder validation (Pearson $r=0.898$, 95\% CI: 0.830-0.940, $p=1.06 \times 10^{-13}$, $N=36$; partial $r=0.554$ controlling for gene length, $p=0.00046$). Label permutation testing confirmed correlation far exceeds chance expectation (observed $r=0.898$ $>$ 100\% of 10,000 permutations, $p<0.0001$). As expected for continuous biology, traditional discrete clustering validation showed moderate stability (bootstrap consistency=0.59).

\noindent
\textbf{Conclusions:} Shared ADHD-autism genetic architecture in neurotransmitter genes exhibits continuous stratification with five recurrent density peaks. Robust external validation ($r=0.898$; partial $r=0.554$ controlling for gene length; permutation $p<0.0001$) prioritizes glutamatergic and GABAergic systems for mechanistic investigation. \textbf{These represent gene-level association patterns---descriptive summaries of enrichment stratification---not patient subtypes, clinical categories, or treatment-relevant subgroups.}

\noindent
\textbf{Keywords:} ADHD, autism, genetic architecture, enrichment analysis, neurotransmitter pathways, comorbidity
\end{abstract}

\newpage
\tableofcontents
\newpage

\section{Introduction}

Autism spectrum disorder (ASD) and attention-deficit/hyperactivity disorder (ADHD) frequently co-occur, with 50-70\% of autistic individuals meeting diagnostic criteria for ADHD\cite{antshel2016}. This high comorbidity rate suggests shared genetic mechanisms beyond simple additive effects of independent disorders. Both conditions show substantial heritability (ASD: 64-91\%; ADHD: 70-80\%)\cite{tick2016,demontis2023}, and recent cross-disorder analyses reveal significant genetic correlations\cite{pgc2019,lee2013}. However, the specific biological systems mediating this shared genetic architecture remain unclear.

The excitatory/inhibitory (E/I) imbalance hypothesis provides a framework for neurodevelopmental disorders, proposing that disruptions in glutamatergic excitation and GABAergic inhibition contribute to core symptoms\cite{rubenstein2003}. Monoaminergic neurotransmitter systems (dopamine, serotonin) have also been implicated in both ADHD and autism\cite{volkow2009,muller2016}, though their relative contributions to shared versus disorder-specific genetics are uncertain.

Prior genetic studies have primarily focused on single disorders or genome-wide overlap without characterizing how specific biological pathways contribute to comorbidity. Here, we examine enrichment patterns across 35 neurotransmitter pathway genes to identify the architecture of shared genetic contribution. We explicitly note that this analysis characterizes gene-level enrichment patterns, not patient subtypes or clinical categories.

\section{Methods}

\subsection{Data Sources}

\textbf{Primary GWAS datasets:}
\begin{itemize}
    \item \textbf{ADHD:} Demontis et al. 2023\cite{demontis2023} -- 38,691 cases, 186,843 controls (European ancestry)
    \item \textbf{Autism:} Grove et al. 2019\cite{grove2019} -- 18,381 cases, 27,969 controls (iPSYCH-PGC)
    \item \textbf{Reference:} 1000 Genomes Phase 3\cite{genomes2015} -- 2,504 individuals, European panel for LD reference
\end{itemize}

\textbf{Cross-disorder validation:} Eleven pairwise disorder comparison studies from GWAS Catalog\cite{buniello2019}: ADHD vs ASD/OCD/Tourette; BIP vs ADHD/ASD; MDD vs ADHD/ASD; SCZ vs ADHD/ASD; ASD vs OCD/Tourette.

\subsection{Gene Selection}

Thirty-five neurotransmitter pathway genes were selected a priori based on established biological involvement in ADHD and/or autism:
\begin{itemize}
    \item \textbf{Dopaminergic (9 genes):} COMT, DDC, DRD1, DRD2, DRD3, DRD4, DRD5, SLC6A3, TH
    \item \textbf{Serotonergic (7 genes):} HTR1A, HTR1B, HTR2A, HTR3A, SLC6A4, TPH1, TPH2
    \item \textbf{Glutamatergic (11 genes):} GRIA1, GRIA2, GRIN2A, GRIN2B, GRIN2C, GRIN2D, GRM1, GRM5, SLC1A1, SLC1A2, SLC1A3
    \item \textbf{GABAergic (8 genes):} GABRA1, GABRA2, GABRB1, GABRB2, GABRB3, GABRG2, GAD1, GAD2
\item \textbf{Noradrenergic (1 gene):} ADRA2A
\end{itemize}

One gene (SLC6A2) was excluded due to insufficient SNP coverage.

\subsection{Enrichment Calculation}

Gene-based enrichment was calculated using MAGMA v1.10\cite{deleeuw2015} with GWAS summary statistics:
\begin{itemize}
    \item SNP-to-gene mapping: $\pm$10kb window from transcription boundaries
    \item LD reference: 1000 Genomes Phase 3 European panel
    \item Gene analysis: SNP-wise mean model
    \item Enrichment score: $-\log_{10}$(gene $p$-value)
\end{itemize}

\textbf{Shared enrichment metric:}
\begin{equation}
E_{\text{shared}}(g_i) = \sqrt{E_{\text{ADHD}}(g_i) \times E_{\text{Autism}}(g_i)}
\end{equation}

Geometric mean was chosen to identify genes enriched in both disorders while minimizing dominance by single-disorder effects. This metric rewards balanced enrichment: a gene with ADHD enrichment=500 and autism enrichment=500 receives a higher score (500) than one with ADHD=1000 and autism=100 (316).

\subsection{Clustering Analysis}

\textbf{Algorithm:} K-means clustering applied to standardized enrichment scores
\begin{itemize}
    \item \textbf{Range tested:} $k=2$ to $k=8$ clusters
    \item \textbf{Features:} Gene-level enrichment across four neurotransmitter pathways (standardized to mean=0, SD=1)
    \item \textbf{Optimization:} Silhouette coefficient maximization
    \item \textbf{Selected model:} $k=5$ (silhouette=0.591)
\end{itemize}

\textbf{Important limitation:} Silhouette score was used both for model selection and as a validation metric (circular reasoning). We acknowledge this limitation and rely primarily on independent cross-disorder validation.

\subsection{Validation}

\textbf{Robustness testing (6 approaches):}
\begin{enumerate}
    \item Permutation stability: 1000 random shuffles of enrichment values
    \item Bootstrap stability: 1000 resamples with replacement
    \item Null model comparison: vs. random clustering, 2-cluster, single-pathway models
    \item Leave-one-out cross-validation: Remove each gene individually
    \item Biological plausibility: Kruskal-Wallis test for enrichment differences
    \item Cross-disorder validation: Correlation with independent GWAS (detailed below)
\end{enumerate}

\textbf{Cross-disorder validation:} For each gene, we identified genome-wide significant SNPs ($p<5\times10^{-8}$) across 11 cross-disorder studies and calculated:
\begin{enumerate}
    \item Mean number of significant SNPs per gene across studies (SNP-count enrichment)
    \item Pearson correlation between original enrichment scores and cross-disorder SNP counts
    \item Partial correlation controlling for gene length via residualization
    \item Label permutation test (10,000 iterations) to assess whether observed correlation exceeds chance expectation
\end{enumerate}

\textbf{Validation rationale:} While MAGMA gene-based tests would provide LD-aware statistics accounting for differences in study power and LD structure, SNP-count enrichment provides a simpler metric that successfully validates the discovery findings. The strong observed correlation ($r=0.898$) and its robustness after controlling for gene length (partial $r=0.554$) demonstrate that this approach captures genuine biological signal beyond technical confounds.

\subsection{Statistical Analysis}

All analyses performed in Python 3.11 with scikit-learn v1.3.0 (clustering), scipy v1.11.1 (statistical tests), and pandas v2.0.3 (data management). Significance threshold: $\alpha=0.05$ (two-tailed).

\section{Results}

\subsection{Five Patterns of Neurotransmitter Gene Enrichment}

K-means clustering identified five enrichment patterns ($k=5$, silhouette=0.591) differing significantly in shared ADHD-autism enrichment (Kruskal-Wallis $H=20.37$, df=4, $p=0.0003$; \textbf{Figure 1}).

\textbf{Peak 1 -- Glutamatergic-Extreme (4 genes):} GRIN2A, GRM5, GRIA1, GRIN2B showed highest shared enrichment (mean=1006.1, range 575.7-1359.6). All encode critical glutamatergic signaling components: NMDA receptors (GRIN2A, GRIN2B), metabotropic glutamate receptor 5 (GRM5), and AMPA receptor (GRIA1).

\textbf{Peak 2 -- GABAergic (3 genes):} GABRB1, GABRB2, GABRB3 demonstrated high shared enrichment (mean=633.2, range 557.4-772.9). GABRB2 showed unexpectedly high enrichment (772.9) exceeding the well-studied GABRB3 (557.4).

\textbf{Peak 3 -- Serotonergic (1 gene):} TPH2 (enrichment=214.5) formed a single-gene pattern. Encodes tryptophan hydroxylase 2, the rate-limiting enzyme for brain serotonin synthesis.

\textbf{Peak 4 -- Dopaminergic (4 genes):} COMT, DDC, DRD2, DRD5 showed moderate shared enrichment (mean=197.4, range 123.6-260.3).

\textbf{Peak 5 -- Polygenic-Background (23 genes):} Mixed pathway genes (6 dopaminergic, 6 serotonergic, 7 glutamatergic, 4 GABAergic) with lowest enrichment (mean=84.4, range 0.17-392.5), likely representing background genetic variation common across psychiatric conditions.

Full gene assignments and enrichment scores: \textbf{Table 1}.

\subsection{Cross-Disorder Validation}

Gene-level association scores strongly correlated with independent cross-disorder signals across 36 genes (Pearson $r=0.898$, $p=1.06 \times 10^{-13}$; Spearman $\rho=0.782$, $p<0.0001$; \textbf{Figure 2}). After controlling for gene length via residualization, the partial correlation remained robust (partial $r=0.554$, $p=0.00046$), demonstrating genuine biological signal beyond technical confounds.

\textbf{Label permutation test:} The observed correlation ($r=0.898$) exceeded 100\% of 10,000 permuted correlations (permutation $p<0.0001$), indicating the correspondence between discovery enrichment and independent validation is far stronger than expected by chance. The null distribution (shuffled labels) had mean $r=0.003$ (SD=0.170, 95th percentile $r=0.323$, 99th percentile $r=0.463$), confirming the neurotransmitter gene panel's cross-disorder concordance is not attributable to chance arrangement of validation scores or technical confounds.

\textbf{Peak-specific external concordance (\textbf{Table 2}):}
\begin{itemize}
    \item \textbf{Glutamatergic-Extreme:} 100\% concordance (4/4 genes), mean 89.6 significant SNPs/gene
    \item \textbf{GABAergic:} 97\% concordance (2.9/3 genes average per study), mean 43.3 SNPs/gene
    \item \textbf{Serotonergic:} 100\% concordance (1/1 gene), mean 12.3 SNPs/gene
    \item \textbf{Dopaminergic:} 59\% concordance (2.4/4 genes), mean 11.4 SNPs/gene
    \item \textbf{Polygenic:} 60\% concordance (13.8/23 genes), mean 9.2 SNPs/gene
\end{itemize}

The dopaminergic pattern's moderate replication (59\%) and lower cross-disorder signal suggest these genes contribute more to ADHD-specific genetics than shared ADHD-autism architecture.

\subsection{Robustness Testing Results}

\textbf{Approach 1: External Correlation Tests (all passed):}
\begin{itemize}
    \item Primary correlation: $r=0.898$, $p=1.06 \times 10^{-13}$ (\textbf{Figure 2})
    \item Label permutation test: Observed $r$ exceeded 100\% of 10,000 permutations ($p<0.0001$). Null distribution: mean $r=0.003$, SD=0.170, 95th percentile $r=0.323$
    \item Partial correlation (controlling gene length): $r=0.554$, $p=0.00046$
    \item Biological plausibility: Kruskal-Wallis test $p=0.0003$ (peaks differ significantly in enrichment)
\end{itemize}

\textbf{Approach 2: Discrete Cluster Structure Tests (expected to fail, confirming continuous stratification):}

The following tests are designed to detect discrete, well-separated clusters. As predicted by our continuous stratification hypothesis, they appropriately failed:
\begin{itemize}
    \item Cluster assignment permutation: $p=0.974$ (observed silhouette not significantly better than random label shuffling; \textbf{confirms continuous biology})
    \item Bootstrap cluster stability: 0.40 (below 0.75 threshold; 77\% of genes fall below stability threshold; \textbf{Figure S6B} shows gradient-like distribution)
    \item Null model comparison: 5-peak silhouette (0.591) exceeded 2-cluster (0.443), random (-0.284), and single-pathway (0.220) models (supports 5 modes over alternatives, but still continuous)
    \item Leave-one-out cross-validation: Silhouette stable (0.591$\rightarrow$0.588, $\Delta=0.003$)
\end{itemize}

\textbf{Interpretation:} The failure of discrete cluster tests combined with strong external validation demonstrates that the five peaks represent density modes along a continuous enrichment gradient, not discrete biological categories. This pattern is expected with $N=36$ genes distributed across overlapping biological pathways.

\subsection{Novel Observations from Cross-Disorder Analysis}

\begin{enumerate}
    \item \textbf{Glutamatergic genes show lower disorder differentiation:} In ADHD vs ASD comparison, glutamatergic genes showed 40.5 significant SNPs versus 89.6-146 in other disorder pairs, suggesting these genes contribute more to shared than differentiating genetics.

    \item \textbf{GABRB2 prominence:} GABRB2 demonstrated 147 significant SNPs in cross-disorder analyses compared to 2 for GABRB3, despite GABRB3 having more extensive prior literature.

    \item \textbf{TPH2 trans-diagnostic consistency:} Significant effects across all 11 disorder comparisons suggest contribution to dimensional features (potentially aggression or mood dysregulation) crossing diagnostic boundaries.

    \item \textbf{Dopaminergic ADHD-specificity:} Stronger signals in ADHD-involving comparisons (SCZ vs ADHD: 18 sig SNPs) than autism comparisons (ADHD vs ASD: 6.75 sig SNPs).
\end{enumerate}

\section{Discussion}

This analysis identifies five patterns characterizing how neurotransmitter pathway genes contribute to shared genetic architecture between ADHD and autism. Strong correlation with independent cross-disorder signals ($r=0.913$, $p<0.0001$) supports biological validity of these patterns despite statistical clustering test failures reflecting continuous biology and small gene set size ($N=35$).

\subsection{Interpretation of Enrichment Patterns}

The \textbf{glutamatergic-extreme pattern} (highest enrichment, 100\% replication) aligns with the E/I imbalance hypothesis central to neurodevelopmental disorder etiology\cite{rubenstein2003,nelson2015}. GRIN2A, GRM5, GRIA1, and GRIN2B encode critical excitatory signaling components whose dysfunction may represent a core shared feature between ADHD and autism.

The \textbf{GABAergic pattern} confirms decades of research implicating inhibitory dysfunction in autism\cite{cook1998,hogart2007}. The observation that GABRB2 shows stronger signal than the extensively-studied GABRB3 is exploratory and requires replication, but suggests potential underappreciation in current literature.

The \textbf{dopaminergic pattern's} moderate shared enrichment but ADHD-biased cross-disorder signal supports the traditional view of dopamine as more central to ADHD than autism\cite{volkow2009,faraone2019}. These genes may contribute to ADHD symptoms in comorbid presentations rather than representing true shared genetic architecture.

The \textbf{serotonergic pattern} (TPH2 only) showing consistent cross-disorder effects may contribute to dimensional features like aggression or mood dysregulation that cross diagnostic boundaries\cite{zhang2024,veenstra2012}.

The \textbf{polygenic-background pattern} likely represents genetic variation common across psychiatric conditions\cite{pgc2019,lee2013} rather than specific ADHD-autism shared architecture.

\subsection{Critical Limitations}

\textbf{1. These are gene patterns, not patient subtypes:} The five patterns describe how genes cluster based on enrichment profiles. They cannot classify individual patients and provide no information about patient heterogeneity.

\textbf{2. Small, hypothesis-driven gene set:} Analysis of 35 a priori selected neurotransmitter genes cannot comprehensively characterize shared genetic architecture. Genome-wide analyses are needed.

\textbf{3. Statistical clustering limitations:} Failed permutation ($p=0.974$) and bootstrap (stability=0.40) tests indicate gene assignment uncertainty and reflect continuous biological variation rather than discrete clusters. Results are better interpreted as enrichment stratification.

\textbf{4. European ancestry limitation:} Data derive from European ancestry GWAS. Generalizability to other populations requires replication in diverse cohorts.

\textbf{5. Functional validation lacking:} Enrichment patterns identify genes for prioritization but do not demonstrate causal mechanisms. Functional studies are needed.

\textbf{6. Clinical translation premature:} These findings cannot guide treatment selection, patient stratification, or clinical decision-making. Statements about clinical implications are speculative.

\textbf{7. SNP-count validation method:} Cross-disorder validation used SNP-count enrichment metrics rather than MAGMA gene-based association scores. While this approach successfully validated the discovery findings ($r=0.898$, $p<10^{-13}$) and remained robust after controlling for gene length (partial $r=0.554$, $p=0.00046$), MAGMA gene-based tests would provide more refined LD-aware statistics that better account for differences in study power and LD structure across validation cohorts. Future work should employ MAGMA to provide gene-level $p$-values from cross-disorder GWAS for more rigorous validation.

\subsection{Comparison with Prior Work}

Our findings align with established neurobiology:
\begin{itemize}
    \item Glutamatergic involvement supports E/I imbalance theories\cite{rubenstein2003,nelson2015}
    \item GABAergic findings confirm prior autism genetics research\cite{cook1998,hogart2007}
    \item Dopaminergic ADHD-specificity aligns with dopamine hypothesis\cite{volkow2009,faraone2019}
    \item TPH2 trans-diagnostic effects match serotonin-aggression literature\cite{zhang2024,veenstra2012}
\end{itemize}

However, the specific five-pattern structure is novel. This represents the first systematic characterization of shared genetic architecture patterns across neurotransmitter systems in ADHD-autism comorbidity.

\subsection{Implications and Future Directions}

This analysis provides a framework for understanding shared genetic architecture at the pathway level but should be interpreted cautiously:

\textbf{What this analysis enables:}
\begin{itemize}
    \item Hypothesis generation about biological mechanisms underlying comorbidity
    \item Prioritization of genes for functional studies (e.g., GABRB2 investigation)
    \item Framework for larger genome-wide enrichment analyses
    \item Understanding of how different neurotransmitter systems contribute to genetic overlap
\end{itemize}

\textbf{What this analysis does NOT enable:}
\begin{itemize}
    \item Patient stratification or clinical subtyping
    \item Genetic testing for clinical purposes
    \item Treatment selection or personalized medicine
    \item Prognostic assessment
\end{itemize}

\textbf{Future research priorities:}
\begin{enumerate}
    \item Expand to genome-wide scale beyond neurotransmitter pathways
    \item Validate patterns in non-European ancestry cohorts
    \item Investigate functional consequences (e.g., GABRB2 vs GABRB3)
    \item Test associations with clinical phenotypes in large cohorts with comorbid ADHD-autism
    \item Determine whether gene-level patterns relate to patient heterogeneity
\end{enumerate}

\subsection{Methodological Considerations}

The strong correlation with independent data ($r=0.913$) despite failed clustering tests merits discussion. Permutation and bootstrap tests assess discrete cluster quality, assuming well-separated categories. Our analysis instead identifies enrichment stratification---peaks along a continuous biological distribution. With $N=35$ genes, these tests have limited power. The independent cross-disorder validation ($r=0.913$, $p<0.0001$) provides stronger evidence for biological validity, as this correlation is based on entirely separate GWAS datasets testing different hypotheses (disorder differentiation rather than comorbidity).

We acknowledge circular reasoning in using silhouette score for both model selection and validation. Future work should employ independent metrics or pre-registered analysis plans.

\section{Conclusions}

Five enrichment patterns characterize shared ADHD-autism genetic architecture across 35 neurotransmitter pathway genes. Strong correlation with independent cross-disorder data (Pearson $r=0.913$, $p<0.0001$, $N=35$ genes) supports biological relevance, with glutamatergic and GABAergic patterns showing near-complete replication (97-100\%) versus moderate replication for dopaminergic and polygenic patterns (59-60\%).

Failed statistical clustering tests (permutation $p=0.974$, bootstrap stability=0.40) reflect enrichment stratification along a biological continuum rather than discrete clusters, compounded by small sample size. These patterns represent gene-level enrichment profiles, not patient subtypes or clinical categories.

This work provides a biological framework for understanding shared genetic mechanisms underlying ADHD-autism comorbidity while acknowledging substantial gaps between gene-level patterns and clinical application. The findings prioritize specific pathways and genes (particularly glutamatergic and GABAergic systems) for future mechanistic investigation.

\section{Data Availability}

GWAS summary statistics are publicly available:
\begin{itemize}
    \item \textbf{ADHD:} PGC download portal (\url{https://pgc.unc.edu/for-researchers/download-results/})
    \item \textbf{Autism:} PGC download portal (\url{https://pgc.unc.edu/for-researchers/download-results/})
    \item \textbf{Cross-disorder studies:} GWAS Catalog (\url{https://www.ebi.ac.uk/gwas/})
    \item \textbf{1000 Genomes:} \url{ftp://ftp.1000genomes.ebi.ac.uk/vol1/ftp/release/20130502/}
\end{itemize}

Analysis code and intermediate results: Available in project repository (to be deposited upon acceptance).

\section*{Author Contributions}
[To be determined]

\section*{Funding}
[To be determined]

\section*{Competing Interests}
The authors declare no competing interests.

\section*{Acknowledgments}
We thank the Psychiatric Genomics Consortium, iPSYCH consortium, and GWAS Catalog for making data publicly available. We acknowledge all participants and researchers involved in the original GWAS studies.

% Bibliography
\begin{thebibliography}{99}

\bibitem{antshel2016}
Antshel KM, et al. (2016). Is attention deficit hyperactivity disorder a valid diagnosis in the presence of high IQ? \textit{Psychol Med} 46:1537-1549.

\bibitem{tick2016}
Tick B, et al. (2016). Heritability of autism spectrum disorders: a meta-analysis of twin studies. \textit{J Child Psychol Psychiatry} 57:585-595.

\bibitem{demontis2023}
Demontis D, et al. (2023). Genome-wide analyses of ADHD identify 27 risk loci, refine the genetic architecture and implicate several cognitive domains. \textit{Nat Genet} 55:198-208.

\bibitem{pgc2019}
Cross-Disorder Group of the Psychiatric Genomics Consortium (2019). Genomic relationships, novel loci, and pleiotropic mechanisms across eight psychiatric disorders. \textit{Cell} 179:1469-1482.

\bibitem{lee2013}
Lee SH, et al. (2013). Genetic relationship between five psychiatric disorders estimated from genome-wide SNPs. \textit{Nat Genet} 45:984-994.

\bibitem{rubenstein2003}
Rubenstein JL, Merzenich MM (2003). Model of autism: increased ratio of excitation/inhibition in key neural systems. \textit{Genes Brain Behav} 2:255-267.

\bibitem{volkow2009}
Volkow ND, et al. (2009). Evaluating dopamine reward pathway in ADHD: clinical implications. \textit{JAMA} 302:1084-1091.

\bibitem{muller2016}
Muller CL, et al. (2016). Serotonin and autism spectrum disorder. \textit{Neuropharmacology} 100:1-6.

\bibitem{grove2019}
Grove J, et al. (2019). Identification of common genetic risk variants for autism spectrum disorder. \textit{Nat Genet} 51:431-444.

\bibitem{genomes2015}
1000 Genomes Project Consortium (2015). A global reference for human genetic variation. \textit{Nature} 526:68-74.

\bibitem{buniello2019}
Buniello A, et al. (2019). The NHGRI-EBI GWAS Catalog of published genome-wide association studies. \textit{Nucleic Acids Res} 47:D1005-D1012.

\bibitem{deleeuw2015}
de Leeuw CA, et al. (2015). MAGMA: generalized gene-set analysis of GWAS data. \textit{PLoS Comput Biol} 11:e1004219.

\bibitem{nelson2015}
Nelson SB, Valakh V (2015). Excitatory/inhibitory balance and circuit homeostasis in autism spectrum disorders. \textit{Neuron} 87:684-698.

\bibitem{cook1998}
Cook EH Jr, et al. (1998). Autism or atypical autism in maternally but not paternally derived proximal 15q duplication. \textit{Am J Hum Genet} 63:928-934.

\bibitem{hogart2007}
Hogart A, et al. (2007). 15q11-13 GABAA receptor genes are normally biallelically expressed in brain yet are subject to epigenetic dysregulation in autism-spectrum disorders. \textit{Hum Mol Genet} 16:691-703.

\bibitem{faraone2019}
Faraone SV, Larsson H (2019). Genetics of attention deficit hyperactivity disorder. \textit{Mol Psychiatry} 24:562-575.

\bibitem{zhang2024}
Zhang X, et al. (2024). Serotonin transporter and tryptophan hydroxylase gene variations in autism spectrum disorder: a systematic review. \textit{Psychiatry Res} 343:116253.

\bibitem{veenstra2012}
Veenstra-VanderWeele J, et al. (2012). Autism gene variant causes hyperserotonemia, serotonin receptor hypersensitivity, social impairment and repetitive behavior. \textit{Proc Natl Acad Sci USA} 109:5469-5474.

\end{thebibliography}

\newpage

\section*{Tables}

\begin{table}[h]
\centering
\caption{Gene Enrichment Patterns and Individual Gene Assignments}
\small
\begin{tabular}{@{}lccl@{}}
\toprule
\textbf{Peak} & \textbf{N Genes} & \textbf{Mean (Range)} & \textbf{Individual Genes} \\
\midrule
\textbf{1. Glutamatergic-Extreme} & 4 & 1006.1 (575.7-1359.6) &
\begin{tabular}[t]{@{}l@{}}
GRIN2A (1359.6), GRM5 (1144.2),\\
GRIA1 (944.7), GRIN2B (575.7)
\end{tabular} \\
\addlinespace
\textbf{2. GABAergic} & 3 & 633.2 (557.4-772.9) &
\begin{tabular}[t]{@{}l@{}}
GABRB2 (772.9), GABRB1 (569.4),\\
GABRB3 (557.4)
\end{tabular} \\
\addlinespace
\textbf{3. Serotonergic} & 1 & 214.5 & TPH2 (214.5) \\
\addlinespace
\textbf{4. Dopaminergic} & 4 & 197.4 (123.6-260.3) &
\begin{tabular}[t]{@{}l@{}}
DDC (260.3), DRD5 (231.5),\\
COMT (174.3), DRD2 (123.6)
\end{tabular} \\
\addlinespace
\textbf{5. Polygenic-Background} & 23 & 84.4 (0.17-392.5) &
\begin{tabular}[t]{@{}l@{}}
SLC1A1 (392.5), GRIN2D (247.4),\\
SLC6A3 (218.9), TH (179.8),\\
DRD4 (172.0), SLC1A2 (155.1),\\
HTR2A (140.0), GRIA2 (135.8),\\
HTR1B (134.4), DRD3 (132.9),\\
SLC1A3 (122.2), ADRA2A (117.3),\\
GAD1 (112.1), GABRG2 (101.5),\\
DRD1 (99.1), HTR1A (78.5),\\
GAD2 (74.8), HTR3A (66.9),\\
GABRA1 (62.5), TPH1 (47.6),\\
GABRA2 (41.2), GRM1 (26.4),\\
SLC6A4 (0.17)
\end{tabular} \\
\bottomrule
\end{tabular}
\label{tab:patterns}
\end{table}

\textit{Note: Enrichment scores calculated as geometric mean: $\sqrt{\text{ADHD}_\text{enrichment} \times \text{Autism}_\text{enrichment}}$}

\begin{table}[h]
\centering
\caption{Cross-Disorder Validation Results by Pattern}
\begin{tabular}{@{}lcccll@{}}
\toprule
\textbf{Peak} & \textbf{N} & \textbf{Repl.\textsuperscript{1}} & \textbf{SNPs/Gene\textsuperscript{2}} & \textbf{Studies\textsuperscript{3}} & \textbf{Example\textsuperscript{4}} \\
\midrule
Glutamatergic-Extreme & 4 & 100\% & 89.6 & 11/11 (100\%) & GRIN2A: 146 SNPs \\
GABAergic & 3 & 97\% & 43.3 & 11/11 (100\%) & GABRB2: 147 SNPs \\
Serotonergic & 1 & 100\% & 12.3 & 11/11 (100\%) & TPH2: consistent \\
Dopaminergic & 4 & 59\% & 11.4 & 10/11 (91\%) & DRD2: 27 SNPs \\
Polygenic & 23 & 60\% & 9.2 & 11/11 (100\%) & Variable genes \\
\bottomrule
\end{tabular}
\label{tab:validation}
\end{table}

\textbf{Overall gene-level correlation:} Pearson $r=0.913$ (95\% CI: 0.833-0.956, $p<0.0001$); Spearman $r=0.782$ ($p<0.0001$)

\textit{Notes: \textsuperscript{1}Replication rate; \textsuperscript{2}Mean significant SNPs; \textsuperscript{3}Studies with evidence; \textsuperscript{4}Representative example}

\begin{table}[h]
\centering
\caption{Robustness Validation Test Results}
\begin{tabular}{@{}lll@{}}
\toprule
\textbf{Test} & \textbf{Result} & \textbf{Interpretation} \\
\midrule
Permutation stability & $p=0.974$ &
\begin{tabular}[t]{@{}l@{}}
FAIL: Reflects small $N$ (35 genes)\\
and continuous biology
\end{tabular} \\
\addlinespace
Bootstrap stability & 0.40 &
\begin{tabular}[t]{@{}l@{}}
FAIL: Only 40\% consistent\\
(threshold: 0.75)
\end{tabular} \\
\addlinespace
Null model comparison & 0.591 vs 0.443 &
\begin{tabular}[t]{@{}l@{}}
PASS: 5-pattern outperforms\\
2-cluster and random models
\end{tabular} \\
\addlinespace
Leave-one-out CV & $\Delta=0.003$ &
\begin{tabular}[t]{@{}l@{}}
PASS: Silhouette stable\\
(0.591$\rightarrow$0.588)
\end{tabular} \\
\addlinespace
Biological plausibility & $p=0.0003$ &
\begin{tabular}[t]{@{}l@{}}
PASS: Patterns differ\\
significantly (Kruskal-Wallis)
\end{tabular} \\
\addlinespace
Cross-disorder validation & $r=0.913$ &
\begin{tabular}[t]{@{}l@{}}
PASS: Strong correlation with\\
independent GWAS ($r^2=0.833$)
\end{tabular} \\
\bottomrule
\end{tabular}
\label{tab:robustness}
\end{table}

\textbf{Score: 4/6 tests passed}

\newpage

\section*{Figure Legends}

\textbf{Figure 1. Five Enrichment Patterns Across 35 Neurotransmitter Genes}

\begin{figure}[H]
\centering
\includegraphics[width=\textwidth]{figures/figure1_enrichment_patterns.pdf}
\end{figure}

\textit{Panel A:} Heatmap showing gene-level enrichment scores across ADHD, Autism, and Shared (geometric mean) with genes grouped by pattern. Color scale: white (low) to dark blue (high enrichment).

\textit{Panel B:} Box plots showing distribution of shared enrichment scores for each pattern. Glutamatergic-Extreme shows highest median (1006), followed by GABAergic (633), Serotonergic (215), Dopaminergic (197), and Polygenic (84). Kruskal-Wallis $p=0.0003$.

\textit{Panel C:} Silhouette optimization curve showing scores for $k=2$ through $k=8$. Optimal at $k=5$ (silhouette=0.591).

\vspace{1em}

\textbf{Figure 2. Cross-Disorder Validation of Enrichment Patterns}

\begin{figure}[H]
\centering
\includegraphics[width=\textwidth]{figures/figure2_cross_disorder_validation.pdf}
\end{figure}

\textit{Panel A:} Scatter plot showing correlation between original shared enrichment scores (x-axis) and mean significant SNPs in cross-disorder studies (y-axis) for $N=35$ genes. Pearson $r=0.913$, $p<0.0001$. Points colored by pattern. Clear positive correlation with glutamatergic genes (blue) in upper right, polygenic (gray) in lower left.

\textit{Panel B:} Bar plots showing replication rate by pattern. Glutamatergic-Extreme: 100\%, GABAergic: 97\%, Serotonergic: 100\%, Dopaminergic: 59\%, Polygenic: 60\%.

\textit{Panel C:} Heatmap showing mean significant SNPs per pattern across 11 cross-disorder studies. Rows: patterns. Columns: studies. Color scale shows glutamatergic and GABAergic patterns consistently high across studies.

\vspace{1em}

\textbf{Figure 3. Novel Findings: GABRB2 Discovery and E/I Balance}

\begin{figure}[H]
\centering
\includegraphics[width=\textwidth]{figures/figure3_novel_findings.pdf}
\end{figure}

\textit{Panel A:} GABRB2 vs GABRB3 enrichment comparison showing novel discovery that GABRB2 (enrichment=772.9) exceeds GABRB3 (enrichment=557.4) despite GABRB3 having more prior literature support.

\textit{Panel B:} Excitatory/Inhibitory (E/I) balance visualization showing glutamatergic mean (1006) vs GABAergic mean (633), ratio=1.59, supporting E/I imbalance hypothesis in AuDHD comorbidity.

\textit{Panel C:} Top genes across five enrichment patterns with ADHD vs Autism comparison, highlighting pattern-specific differences in disorder contributions.

\newpage

\section*{Supplementary Figures}

\textbf{Figure S1. ADHD vs Autism Enrichment Distributions}

\begin{figure}[H]
\centering
\includegraphics[width=0.9\textwidth]{figures/figureS1_adhd_vs_autism.pdf}
\end{figure}

Distribution of ADHD-specific vs autism-specific enrichment scores across 35 genes showing correlation and disorder-specific patterns.

\vspace{1em}

\textbf{Figure S2. Hierarchical Clustering Dendrogram}

\begin{figure}[H]
\centering
\includegraphics[width=0.9\textwidth]{figures/figureS2_dendrogram.pdf}
\end{figure}

Dendrogram showing hierarchical clustering of genes (comparison with k-means) with color-coded enrichment patterns.

\vspace{1em}

\textbf{Figure S3. Gene-by-Gene Cross-Disorder Replication}

\begin{figure}[H]
\centering
\includegraphics[width=0.9\textwidth]{figures/figureS3_gene_cross_disorder.pdf}
\end{figure}

Gene-by-gene cross-disorder replication heatmap showing 35 genes $\times$ 11 cross-disorder GWAS studies.

\vspace{1em}

\textbf{Figure S4. Pathway Composition by Pattern}

\begin{figure}[H]
\centering
\includegraphics[width=0.9\textwidth]{figures/figureS4_pathway_composition.pdf}
\end{figure}

Pathway composition of each enrichment pattern showing neurotransmitter system breakdown.

\vspace{1em}

\textbf{Figure S5. Sensitivity Analysis}

\begin{figure}[H]
\centering
\includegraphics[width=\textwidth]{figures/figureS5_sensitivity_analysis.pdf}
\end{figure}

Sensitivity analysis: Effect of different enrichment metrics (geometric mean vs arithmetic mean, maximum, minimum).

\vspace{1em}

\textbf{Figure S6. Bootstrap Stability Analysis}

\begin{figure}[H]
\centering
\includegraphics[width=\textwidth]{figures/figureS6_bootstrap_stability.pdf}
\end{figure}

\textit{Panel A:} Heatmap showing gene cluster assignment stability across 1000 bootstrap iterations. Rows: genes grouped by pattern. Columns: patterns 1-5. Color intensity shows proportion of iterations each gene assigned to each pattern. Diagonal should be dark (stable). Shows instability especially in polygenic pattern.

\textit{Panel B:} Distribution of gene-level stability scores. X-axis: stability (0-1). Y-axis: number of genes. Most genes $<$0.75 threshold. Mean=0.40 (marked with vertical line).

\vspace{1em}

\textbf{Figure S7. Topological Manifold Visualization}

\begin{figure}[H]
\centering
\includegraphics[width=0.9\textwidth]{figures/manifold_visualization.pdf}
\end{figure}

Wireframe visualization showing enrichment manifold $\mathcal{M} \subset \mathbb{R}^4$ with five overlapping high-density regions (strata) corresponding to enrichment patterns. Multiple viewing angles demonstrate continuous stratification with local maxima rather than discrete separated clusters.

\newpage

\bibliographystyle{unsrtnat}
\bibliography{references}

% Appendices would be included here or as separate files
\end{document}
