\documentclass[12pt,letterpaper]{article}

% Essential packages
\usepackage[utf8]{inputenc}
\usepackage[T1]{fontenc}
\usepackage{lmodern}
\usepackage[margin=1in]{geometry}
\usepackage{setspace}
\onehalfspacing

% Math packages
\usepackage{amsmath,amssymb,amsthm,mathtools}
\usepackage{bm}

% Graphics and tables
\usepackage{graphicx}
\usepackage{booktabs}

% References
\usepackage{hyperref}
\hypersetup{
    colorlinks=true,
    linkcolor=blue,
    citecolor=blue,
    urlcolor=blue
}

% Theorem environments
\newtheorem{theorem}{Theorem}[section]
\newtheorem{lemma}[theorem]{Lemma}
\newtheorem{proposition}[theorem]{Proposition}
\theoremstyle{definition}
\newtheorem{definition}[theorem]{Definition}
\theoremstyle{remark}
\newtheorem{remark}[theorem]{Remark}

\title{\textbf{Appendix A: Mathematical and Topological Framework}}
\author{}
\date{}

\begin{document}

\maketitle

\section{Formal Problem Statement}

\subsection{Gene Enrichment Space}

Let $\mathcal{G} = \{g_1, \ldots, g_N\}$ denote a set of $N$ genes ($N=36$ in our analysis). For each gene $g_i$ and disorder $d \in \{\text{ADHD}, \text{ASD}\}$, we define an enrichment score:

\begin{equation}
E_d(g_i) = -\log_{10}(p_d(g_i))
\end{equation}

where $p_d(g_i)$ is the gene-level $p$-value from MAGMA analysis of disorder $d$.

\subsection{Shared Enrichment Metric}

The shared enrichment for gene $g_i$ across ADHD and autism is defined using the geometric mean:

\begin{equation}
E_{\text{shared}}(g_i) = \sqrt{E_{\text{ADHD}}(g_i) \cdot E_{\text{ASD}}(g_i)}
\end{equation}

\textbf{Justification for geometric mean:}

\begin{enumerate}
    \item \textbf{Balanced contribution}: For two values $a, b$ with $a < b$, the geometric mean $\sqrt{ab}$ is bounded by:
    \begin{equation}
    a \leq \sqrt{ab} \leq \frac{a+b}{2} \leq b
    \end{equation}
    
    \item \textbf{Multiplicative interpretation}: Equivalent to arithmetic mean in log-space:
    \begin{equation}
    \sqrt{ab} = \exp\left(\frac{\log a + \log b}{2}\right)
    \end{equation}
    
    \item \textbf{Penalizes imbalance}: Maximized when $a = b$:
    \begin{equation}
    \frac{\partial}{\partial a}\sqrt{ab}\bigg|_{a=b} = \frac{b}{2\sqrt{ab}} = \frac{1}{2}\sqrt{\frac{b}{a}} \rightarrow \infty \text{ as } a \rightarrow 0
    \end{equation}
\end{enumerate}

\subsection{Feature Space Construction}

Each gene $g_i$ is represented by a feature vector in pathway space:

\begin{equation}
\mathbf{f}(g_i) = \begin{bmatrix}
E_{\text{DA}}(g_i) \\
E_{\text{5HT}}(g_i) \\
E_{\text{Glu}}(g_i) \\
E_{\text{GABA}}(g_i)
\end{bmatrix} \in \mathbb{R}^4
\end{equation}

where subscripts denote dopaminergic (DA), serotonergic (5HT), glutamatergic (Glu), and GABAergic pathways.

\textbf{Standardization}: Features are z-score normalized:
\begin{equation}
\tilde{\mathbf{f}}(g_i) = \frac{\mathbf{f}(g_i) - \boldsymbol{\mu}}{\boldsymbol{\sigma}}
\end{equation}

where $\boldsymbol{\mu}$ and $\boldsymbol{\sigma}$ are component-wise mean and standard deviation across all genes.

\section{Topological Structure of Enrichment Space}

\subsection{Enrichment Manifold}

The set of gene enrichment profiles forms a submanifold $\mathcal{M} \subset \mathbb{R}^4$:

\begin{equation}
\mathcal{M} = \{\tilde{\mathbf{f}}(g_i) : g_i \in \mathcal{G}\}
\end{equation}

\textbf{Ambient space}: $\mathcal{M} \subset \mathbb{R}^4$ with standard Euclidean metric

\textbf{Intrinsic dimension}: Estimated via local PCA or persistent homology

\subsection{Metric Structure}

We endow $\mathcal{M}$ with the induced Euclidean metric:

\begin{equation}
d(\mathbf{f}(g_i), \mathbf{f}(g_j)) = \|\tilde{\mathbf{f}}(g_i) - \tilde{\mathbf{f}}(g_j)\|_2
\end{equation}

This metric satisfies:
\begin{enumerate}
    \item \textbf{Positivity}: $d(\mathbf{f}_i, \mathbf{f}_j) \geq 0$ with equality iff $i = j$
    \item \textbf{Symmetry}: $d(\mathbf{f}_i, \mathbf{f}_j) = d(\mathbf{f}_j, \mathbf{f}_i)$
    \item \textbf{Triangle inequality}: $d(\mathbf{f}_i, \mathbf{f}_k) \leq d(\mathbf{f}_i, \mathbf{f}_j) + d(\mathbf{f}_j, \mathbf{f}_k)$
\end{enumerate}

\subsection{Stratification vs. Clustering}

\textbf{Traditional clustering assumption} (NOT applicable here):
\begin{equation}
\mathcal{M} = \bigsqcup_{k=1}^K \mathcal{C}_k
\end{equation}

where $\mathcal{C}_k$ are disjoint, well-separated components.

\textbf{Our model} (enrichment stratification):
\begin{equation}
\mathcal{M} = \bigcup_{k=1}^K \mathcal{S}_k
\end{equation}

where $\mathcal{S}_k$ are overlapping high-density regions (``strata'') along a continuum.

\textbf{Mathematical distinction}:
\begin{itemize}
    \item Clustering: $\mathcal{C}_i \cap \mathcal{C}_j = \emptyset$ for $i \neq j$
    \item Stratification: $\mathcal{S}_i \cap \mathcal{S}_j \neq \emptyset$ possible
\end{itemize}

\section{K-Means Clustering Algorithm}

\subsection{Objective Function}

K-means minimizes within-cluster sum of squares (WCSS):

\begin{equation}
\underset{\{\mathcal{C}_1, \ldots, \mathcal{C}_K\}}{\arg\min} \sum_{k=1}^K \sum_{g_i \in \mathcal{C}_k} \|\tilde{\mathbf{f}}(g_i) - \boldsymbol{\mu}_k\|^2
\end{equation}

where $\boldsymbol{\mu}_k = \frac{1}{|\mathcal{C}_k|}\sum_{g_i \in \mathcal{C}_k} \tilde{\mathbf{f}}(g_i)$ is the centroid of cluster $k$.

\subsection{Lloyd's Algorithm}

\textbf{Initialization}: Random selection of $K$ initial centroids

\textbf{Iteration} (until convergence):
\begin{enumerate}
    \item \textbf{Assignment step}: For each gene $g_i$, assign to nearest centroid:
    \begin{equation}
    c(g_i) = \underset{k \in \{1,\ldots,K\}}{\arg\min} \|\tilde{\mathbf{f}}(g_i) - \boldsymbol{\mu}_k\|^2
    \end{equation}
    
    \item \textbf{Update step}: Recompute centroids:
    \begin{equation}
    \boldsymbol{\mu}_k \leftarrow \frac{1}{|\mathcal{C}_k|} \sum_{g_i : c(g_i)=k} \tilde{\mathbf{f}}(g_i)
    \end{equation}
\end{enumerate}

\textbf{Convergence}: Guaranteed to converge to a local minimum (not necessarily global).

\textbf{Complexity}: $O(NKdT)$ where $N$ = genes, $K$ = clusters, $d$ = dimensions, $T$ = iterations

\subsection{Cluster Quality Metrics}

\textbf{Silhouette coefficient} for gene $g_i$ in cluster $\mathcal{C}_k$:

\begin{equation}
s(g_i) = \frac{b(g_i) - a(g_i)}{\max\{a(g_i), b(g_i)\}}
\end{equation}

where:
\begin{itemize}
    \item $a(g_i) = \frac{1}{|\mathcal{C}_k| - 1} \sum_{g_j \in \mathcal{C}_k, j \neq i} d(g_i, g_j)$ (mean intra-cluster distance)
    \item $b(g_i) = \min_{l \neq k} \frac{1}{|\mathcal{C}_l|} \sum_{g_j \in \mathcal{C}_l} d(g_i, g_j)$ (mean nearest-cluster distance)
\end{itemize}

\textbf{Average silhouette}:
\begin{equation}
\bar{s} = \frac{1}{N} \sum_{i=1}^N s(g_i)
\end{equation}

\textbf{Interpretation}:
\begin{itemize}
    \item $s(g_i) \approx 1$: Well-clustered (close to own cluster, far from others)
    \item $s(g_i) \approx 0$: On cluster boundary
    \item $s(g_i) < 0$: Likely misassigned
\end{itemize}

\section{Validation Framework}

\subsection{Permutation Test}

\textbf{Null hypothesis} $H_0$: Observed clustering structure is no better than random.

\textbf{Test statistic}: $T = \bar{s}$ (average silhouette score)

\textbf{Procedure}:
\begin{enumerate}
    \item Compute observed $T_{\text{obs}}$ from real data
    \item For $b = 1, \ldots, B$ permutations:
    \begin{itemize}
        \item[(a)] Randomly shuffle enrichment values within each pathway
        \item[(b)] Re-cluster with same $K$
        \item[(c)] Compute $T^{(b)}$
    \end{itemize}
    \item Calculate $p$-value:
    \begin{equation}
    p = \frac{1 + \sum_{b=1}^B \mathbb{1}(T^{(b)} \geq T_{\text{obs}})}{B + 1}
    \end{equation}
\end{enumerate}

\textbf{Our result}: $p = 0.974$ with $B=1000$

\textbf{Interpretation}: Clustering not significantly better than random. However, this test assumes:
\begin{itemize}
    \item Exchangeability of enrichment values (violated: pathway structure)
    \item Discrete well-separated clusters (violated: continuous stratification)
    \item Sufficient sample size (violated: $N=36$)
\end{itemize}

\subsection{Bootstrap Stability}

\textbf{Measure}: Adjusted Rand Index (ARI) between clusterings of bootstrap samples

\textbf{Procedure}:
\begin{enumerate}
    \item For $b = 1, \ldots, B$ bootstrap iterations:
    \begin{itemize}
        \item[(a)] Sample $N$ genes with replacement: $\mathcal{G}^{(b)}$
        \item[(b)] Cluster $\mathcal{G}^{(b)}$ with $K$ clusters
        \item[(c)] Record assignments $\mathbf{c}^{(b)}$
    \end{itemize}
    \item For each gene $g_i$, calculate stability:
    \begin{equation}
    \text{Stability}(g_i) = \frac{1}{B(B-1)/2} \sum_{b < b'} \mathbb{1}(c_i^{(b)} = c_i^{(b')})
    \end{equation}
    \item Average across genes:
    \begin{equation}
    \bar{\text{Stability}} = \frac{1}{N} \sum_{i=1}^N \text{Stability}(g_i)
    \end{equation}
\end{enumerate}

\textbf{Our result}: $\bar{\text{Stability}} = 0.40$ with $B=1000$

\textbf{Interpretation}: Only 40\% stability (threshold: 0.75). Indicates:
\begin{itemize}
    \item Gene assignments uncertain, especially borderline cases
    \item Polygenetic pattern (23 genes) contributes to instability
    \item Reflects continuous nature of enrichment distribution
\end{itemize}

\subsection{Cross-Validation}

\textbf{Leave-one-out cross-validation (LOOCV)}:

For each gene $g_i$:
\begin{enumerate}
    \item Remove $g_i$ from dataset: $\mathcal{G}_{-i} = \mathcal{G} \setminus \{g_i\}$
    \item Cluster $\mathcal{G}_{-i}$ with $K$ clusters
    \item Compute silhouette score $\bar{s}_{-i}$
\end{enumerate}

\textbf{Stability metric}:
\begin{equation}
\Delta_{\text{CV}} = \frac{1}{N} \sum_{i=1}^N |\bar{s} - \bar{s}_{-i}|
\end{equation}

\textbf{Our result}: $\Delta_{\text{CV}} = 0.003$ (mean absolute change)

\textbf{Interpretation}: Clustering highly stable to individual gene removal. Not driven by outliers.

\subsection{Independent Cross-Disorder Validation}

\textbf{Correlation analysis} between original enrichment and independent signals:

For each gene $g_i$, let:
\begin{itemize}
    \item $E_{\text{shared}}(g_i)$ = original shared enrichment
    \item $S_{\text{cross}}(g_i)$ = mean number of genome-wide significant SNPs across 11 cross-disorder studies
\end{itemize}

\textbf{Pearson correlation}:
\begin{equation}
r = \frac{\sum_{i=1}^N (E_{\text{shared}}(g_i) - \bar{E})(S_{\text{cross}}(g_i) - \bar{S})}{\sqrt{\sum_{i=1}^N (E_{\text{shared}}(g_i) - \bar{E})^2} \sqrt{\sum_{i=1}^N (S_{\text{cross}}(g_i) - \bar{S})^2}}
\end{equation}

\textbf{Our result}: $r = 0.898$, $p = 1.06 \times 10^{-13}$ ($N=36$ genes)

\textbf{Spearman rank correlation}:
\begin{equation}
\rho = 1 - \frac{6\sum_{i=1}^N d_i^2}{N(N^2-1)}
\end{equation}

where $d_i$ is the difference in ranks for gene $i$.

\textbf{Our result}: $\rho = 0.782$, $p = 1.06 \times 10^{-13}$

\textbf{Interpretation}: Strong correlation with entirely independent data provides evidence for biological validity despite failed clustering tests.

\section{Statistical Power Analysis}

\subsection{Power for Correlation Detection}

Given $N=36$ genes, the minimum detectable correlation at $\alpha=0.05$, power $=0.80$:

\begin{equation}
r_{\text{min}} = \sqrt{\frac{(z_{\alpha/2} + z_{\beta})^2}{N - 3}}
\end{equation}

where $z_{\alpha/2} = 1.96$ and $z_{\beta} = 0.84$.

\textbf{Calculation}: $r_{\text{min}} = \sqrt{\frac{(1.96 + 0.84)^2}{35 - 3}} = \sqrt{\frac{7.84}{32}} \approx 0.495$

\textbf{Interpretation}: Our observed $r=0.913$ far exceeds minimum detectable effect, indicating high statistical power.

\subsection{Power for Clustering Validation}

For permutation test with $B=1000$ permutations:

\textbf{Minimum detectable effect size} (Cohen's $d$):
\begin{equation}
d_{\text{min}} = \frac{z_{\alpha} + z_{\beta}}{\sqrt{N/2}}
\end{equation}

For $N=36$: $d_{\text{min}} = \frac{1.96 + 0.84}{\sqrt{35/2}} \approx 0.67$

\textbf{Interpretation}: Medium-to-large effects detectable, but small $N$ limits power for subtle clustering patterns.

\section{Computational Complexity}

\subsection{K-Means Algorithm}

\textbf{Time complexity}: $O(NKdT)$
\begin{itemize}
    \item $N = 35$ genes
    \item $K = 5$ clusters
    \item $d = 4$ dimensions (pathways)
    \item $T \approx 100$ iterations (typical convergence)
\end{itemize}

\textbf{Total operations}: $35 \times 5 \times 4 \times 100 = 70,000$

\textbf{Space complexity}: $O(Nd + K) = O(35 \times 4 + 5) = O(145)$

\subsection{Validation Procedures}

\textbf{Permutation test}: $O(BNKdT) = O(1000 \times 70,000) = O(7 \times 10^7)$ operations

\textbf{Bootstrap}: $O(BNKdT) = O(1000 \times 70,000) = O(7 \times 10^7)$ operations

\textbf{Cross-validation}: $O(N^2KdT) = O(35^2 \times 5 \times 4 \times 100) = O(2.45 \times 10^6)$ operations

\textbf{Total computational cost}: Approximately $10^8$ operations, feasible on standard hardware.

\section{Manifold Learning Perspective}

\subsection{Intrinsic Dimensionality}

The effective dimensionality of enrichment manifold $\mathcal{M}$ may be lower than ambient dimension $d=4$.

\textbf{Local PCA estimate}:
For each point $\mathbf{f}(g_i)$ and its $k$-nearest neighbors, compute local covariance matrix $\mathbf{C}_i$.

\textbf{Intrinsic dimension}:
\begin{equation}
\hat{d}_{\text{int}} = \underset{d'}{\arg\max} \left\{ \frac{\sum_{j=1}^{d'} \lambda_j}{\sum_{j=1}^d \lambda_j} \geq 0.95 \right\}
\end{equation}

where $\lambda_1 \geq \ldots \geq \lambda_d$ are eigenvalues of $\mathbf{C}_i$.

\subsection{Geodesic Distance}

True distances on manifold may differ from Euclidean distances in $\mathbb{R}^4$:

\textbf{Geodesic distance} between $g_i$ and $g_j$:
\begin{equation}
d_{\mathcal{M}}(g_i, g_j) = \inf_{\gamma} \int_0^1 \|\gamma'(t)\| \, dt
\end{equation}

where $\gamma : [0,1] \rightarrow \mathcal{M}$ is a path connecting $\mathbf{f}(g_i)$ and $\mathbf{f}(g_j)$.

\textbf{Approximation}: Isomap or diffusion maps could estimate $d_{\mathcal{M}}$.

\section{Alternative Geometric Interpretations}

\subsection{Mixture Model Perspective}

Instead of hard k-means clustering, enrichment could arise from mixture of Gaussians:

\begin{equation}
p(\mathbf{f}(g_i)) = \sum_{k=1}^K \pi_k \mathcal{N}(\mathbf{f}(g_i) | \boldsymbol{\mu}_k, \boldsymbol{\Sigma}_k)
\end{equation}

where $\pi_k$ are mixing proportions, $\boldsymbol{\mu}_k$ are means, $\boldsymbol{\Sigma}_k$ are covariances.

\textbf{Soft assignment} via posterior probability:
\begin{equation}
\gamma_{ik} = \frac{\pi_k \mathcal{N}(\mathbf{f}(g_i) | \boldsymbol{\mu}_k, \boldsymbol{\Sigma}_k)}{\sum_{l=1}^K \pi_l \mathcal{N}(\mathbf{f}(g_i) | \boldsymbol{\mu}_l, \boldsymbol{\Sigma}_l)}
\end{equation}

\textbf{Interpretation}: Would better capture overlapping strata and uncertainty in assignments.

\subsection{Density-Based Perspective}

\textbf{Mode-seeking interpretation}: Enrichment patterns correspond to local maxima of density $p(\mathbf{f})$.

\textbf{Mean-shift algorithm}:
\begin{equation}
\mathbf{m}(\mathbf{x}) = \frac{\sum_{i=1}^N \mathbf{f}(g_i) K(\|\mathbf{x} - \mathbf{f}(g_i)\|)}{\sum_{i=1}^N K(\|\mathbf{x} - \mathbf{f}(g_i)\|)}
\end{equation}

where $K(\cdot)$ is a kernel function.

\textbf{Advantage}: No assumption of spherical clusters or fixed number of patterns.

\section{Limitations and Future Directions}

\subsection{Sample Size}

With $N=36$ genes, statistical power is limited for:
\begin{itemize}
    \item Detecting subtle clustering structure
    \item Robustly estimating cluster boundaries
    \item Validating via resampling methods
\end{itemize}

\textbf{Recommendation}: Expand analysis to genome-wide scale ($N > 10,000$ genes).

\subsection{Feature Engineering}

Current features (pathway-level enrichment) may not capture:
\begin{itemize}
    \item Gene-gene interactions
    \item Tissue-specific expression
    \item Developmental timing
    \item Regulatory networks
\end{itemize}

\textbf{Recommendation}: Integrate multi-omic data (expression, chromatin, protein-protein interactions).

\subsection{Alternative Clustering Methods}

K-means assumes:
\begin{itemize}
    \item Spherical clusters
    \item Similar cluster sizes
    \item Linear separability
\end{itemize}

\textbf{Alternatives to explore}:
\begin{itemize}
    \item Hierarchical clustering with linkage criteria
    \item DBSCAN for density-based patterns
    \item Gaussian mixture models for soft assignments
    \item Spectral clustering for non-convex shapes
\end{itemize}

\subsection{Biological Validation}

Mathematical framework requires experimental validation:
\begin{itemize}
    \item Functional studies of genes within patterns
    \item Patient stratification based on genetic profiles
    \item Treatment response prediction
    \item Animal model phenotypes
\end{itemize}

\end{document}
