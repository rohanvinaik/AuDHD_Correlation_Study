\documentclass[12pt,letterpaper]{article}

% Essential packages
\usepackage[utf8]{inputenc}
\usepackage[T1]{fontenc}
\usepackage{lmodern}
\usepackage[margin=1in]{geometry}
\usepackage{setspace}
\onehalfspacing

% Math packages
\usepackage{amsmath,amssymb}

% Graphics and tables
\usepackage{booktabs}
\usepackage{longtable}

% References
\usepackage{hyperref}
\hypersetup{
    colorlinks=true,
    linkcolor=blue,
    citecolor=blue,
    urlcolor=blue
}

\title{\textbf{Appendix B: Biological Systems and Gene Function}}
\author{}
\date{}

\begin{document}

\maketitle

\section{Neurotransmitter Pathway Gene Functions}

\subsection{Glutamatergic System Genes}

\textbf{GRIN2A} (Glutamate Ionotropic Receptor NMDA Type Subunit 2A)
\begin{itemize}
    \item \textbf{Function}: NMDA receptor subunit; critical for synaptic plasticity and learning
    \item \textbf{Enrichments}: ADHD=1546.0, Autism=1195.7, Shared=1359.6
    \item \textbf{Literature}: Associated with intellectual disability, epilepsy, autism
    \item \textbf{Mechanism}: Loss-of-function mutations impair NMDA-mediated excitatory neurotransmission
\end{itemize}

\textbf{GRM5} (Glutamate Metabotropic Receptor 5)
\begin{itemize}
    \item \textbf{Function}: G-protein coupled mGluR5 receptor; modulates synaptic transmission
    \item \textbf{Enrichments}: ADHD=743.8, Autism=1759.9, Shared=1144.2
    \item \textbf{Literature}: mGluR theory of autism; clinical trials of mGluR5 antagonists
    \item \textbf{Mechanism}: Excessive mGluR5 signaling $\rightarrow$ exaggerated protein synthesis
\end{itemize}

\textbf{GRIA1} (Glutamate Ionotropic Receptor AMPA Type Subunit 1)
\begin{itemize}
    \item \textbf{Function}: AMPA receptor subunit; mediates fast excitatory transmission
    \item \textbf{Enrichments}: ADHD=773.2, Autism=1154.4, Shared=944.7
    \item \textbf{Literature}: Implicated in intellectual disability and autism
    \item \textbf{Mechanism}: Altered AMPA receptor trafficking affects synaptic strength
\end{itemize}

\textbf{GRIN2B} (Glutamate Ionotropic Receptor NMDA Type Subunit 2B)
\begin{itemize}
    \item \textbf{Function}: NMDA receptor subunit; critical for early brain development
    \item \textbf{Enrichments}: ADHD=374.9, Autism=883.9, Shared=575.7
    \item \textbf{Literature}: De novo mutations in autism and schizophrenia
    \item \textbf{Mechanism}: Developmentally regulated; early expression critical for circuit formation
\end{itemize}

\subsection{GABAergic System Genes}

\textbf{GABRB2} (GABA A Receptor Subunit Beta 2)
\begin{itemize}
    \item \textbf{Function}: GABAA receptor subunit; mediates inhibitory neurotransmission
    \item \textbf{Enrichments}: ADHD=587.0, Autism=1017.5, Shared=772.9
    \item \textbf{Novel finding}: Higher cross-disorder signal than GABRB3 (147 vs. 2 significant SNPs)
    \item \textbf{Literature}: Limited prior focus compared to GABRB3
    \item \textbf{Mechanism}: GABAA receptor composition affects inhibitory tone
\end{itemize}

\textbf{GABRB3} (GABA A Receptor Subunit Beta 3)
\begin{itemize}
    \item \textbf{Function}: GABAA receptor subunit; part of 15q11-q13 autism locus
    \item \textbf{Enrichments}: ADHD=407.7, Autism=761.9, Shared=557.4
    \item \textbf{Literature}: Extensively studied; maternal duplications in autism
    \item \textbf{Mechanism}: Reduced GABAA signaling $\rightarrow$ decreased inhibition
\end{itemize}

\textbf{GABRB1} (GABA A Receptor Subunit Beta 1)
\begin{itemize}
    \item \textbf{Function}: GABAA receptor subunit; widespread CNS expression
    \item \textbf{Enrichments}: ADHD=694.4, Autism=466.8, Shared=569.4
    \item \textbf{Literature}: Less studied than GABRB2/3 in neurodevelopmental disorders
    \item \textbf{Mechanism}: Contributes to inhibitory receptor diversity
\end{itemize}

\subsection{Serotonergic System Genes}

\textbf{TPH2} (Tryptophan Hydroxylase 2)
\begin{itemize}
    \item \textbf{Function}: Rate-limiting enzyme for brain serotonin synthesis
    \item \textbf{Enrichments}: ADHD=192.4, Autism=239.2, Shared=214.5
    \item \textbf{Literature}: Associated with aggression, autism, mood disorders
    \item \textbf{Mechanism}: Reduced TPH2 function $\rightarrow$ decreased serotonin synthesis
    \item \textbf{Trans-diagnostic}: Significant across all 11 cross-disorder comparisons
\end{itemize}

\subsection{Dopaminergic System Genes}

\textbf{COMT} (Catechol-O-Methyltransferase)
\begin{itemize}
    \item \textbf{Function}: Degrades catecholamines (dopamine, norepinephrine)
    \item \textbf{Enrichments}: ADHD=156.1, Autism=194.5, Shared=174.3
    \item \textbf{Literature}: Val158Met polymorphism affects dopamine clearance
    \item \textbf{Mechanism}: Lower COMT activity $\rightarrow$ increased prefrontal dopamine
\end{itemize}

\textbf{DDC} (DOPA Decarboxylase)
\begin{itemize}
    \item \textbf{Function}: Converts L-DOPA to dopamine
    \item \textbf{Enrichments}: ADHD=177.0, Autism=382.8, Shared=260.3
    \item \textbf{Mechanism}: Critical for dopamine biosynthesis
\end{itemize}

\textbf{DRD2} (Dopamine Receptor D2)
\begin{itemize}
    \item \textbf{Function}: D2-class dopamine receptor; target of antipsychotics
    \item \textbf{Enrichments}: ADHD=152.9, Autism=99.9, Shared=123.6
    \item \textbf{Literature}: Implicated in ADHD; stimulant medications increase dopamine
    \item \textbf{Mechanism}: Reduced D2 signaling associated with ADHD symptoms
\end{itemize}

\textbf{DRD5} (Dopamine Receptor D5)
\begin{itemize}
    \item \textbf{Function}: D1-class dopamine receptor; high affinity for dopamine
    \item \textbf{Enrichments}: ADHD=200.4, Autism=267.6, Shared=231.5
    \item \textbf{Literature}: Polymorphisms associated with ADHD
    \item \textbf{Mechanism}: Modulates prefrontal cortex function
\end{itemize}

\section{Pathway-Level Biology}

\subsection{Excitatory/Inhibitory Balance}

\textbf{E/I Ratio Hypothesis} (Rubenstein \& Merzenich 2003):
\begin{equation}
\text{E/I ratio} = \frac{\text{Glutamatergic excitation}}{\text{GABAergic inhibition}}
\end{equation}

\textbf{Predictions}:
\begin{itemize}
    \item Increased E/I $\rightarrow$ hyperexcitability, seizures, sensory hypersensitivity
    \item Decreased E/I $\rightarrow$ reduced plasticity, cognitive impairment
\end{itemize}

\textbf{Our findings support}:
\begin{itemize}
    \item Glutamatergic pattern (highest enrichment) $\rightarrow$ excessive excitation
    \item GABAergic pattern (high enrichment) $\rightarrow$ reduced inhibition
    \item Combined effect: Elevated E/I ratio in AuDHD
\end{itemize}

\textbf{Evidence from our data}:
\begin{itemize}
    \item Glutamatergic mean enrichment (1006) $>$ GABAergic (633) in absolute terms
    \item Both patterns show $>$97\% cross-disorder replication
    \item Suggests E/I imbalance is core shared feature
\end{itemize}

\subsection{Monoamine Modulation}

\textbf{Dopamine}: Attention, motivation, reward processing
\begin{itemize}
    \item Primary ADHD neurotransmitter
    \item Dopaminergic pattern shows only 59\% replication $\rightarrow$ ADHD-specific
    \item Lower shared enrichment (197) than glutamate/GABA patterns
\end{itemize}

\textbf{Serotonin}: Mood, aggression, social behavior
\begin{itemize}
    \item TPH2 pattern shows 100\% replication across disorders
    \item Moderate enrichment (215)
    \item Trans-diagnostic role (aggression, mood dysregulation)
\end{itemize}

\subsection{Pathway Interactions}

Neurotransmitter systems do not operate in isolation:

\textbf{Glutamate-GABA reciprocity}:
\begin{itemize}
    \item Excitatory-inhibitory neurons form local circuits
    \item Glutamate can drive feedforward inhibition via GABAergic interneurons
    \item Disruption of either system affects E/I balance
\end{itemize}

\textbf{Monoamine modulation of E/I}:
\begin{itemize}
    \item Dopamine modulates prefrontal glutamate release
    \item Serotonin regulates GABAergic interneuron activity
    \item Complex interactions make prediction difficult
\end{itemize}

\section{Developmental Trajectories}

\subsection{Temporal Dynamics}

Gene expression and function change across development:

\textbf{Early development (prenatal-early postnatal)}:
\begin{itemize}
    \item GRIN2B expression peaks early; critical for synapse formation
    \item GABAergic genes establish inhibitory circuits
    \item Disruption $\rightarrow$ permanent circuit alterations
\end{itemize}

\textbf{Childhood}:
\begin{itemize}
    \item GRIN2A expression increases; replaces GRIN2B in mature synapses
    \item Dopaminergic system matures; relevant for ADHD symptom emergence
    \item Critical period for intervention
\end{itemize}

\textbf{Adolescence}:
\begin{itemize}
    \item Prefrontal cortex maturation
    \item Serotonergic system changes (mood, aggression regulation)
    \item ADHD symptoms may persist or remit
\end{itemize}

\subsection{Critical Periods}

\textbf{Hypothesis}: Genetic variants in glutamatergic/GABAergic genes exert maximal effects during critical periods when:
\begin{enumerate}
    \item Gene expression is highest
    \item Circuits are being established (high plasticity)
    \item Compensatory mechanisms not yet developed
\end{enumerate}

\textbf{Implication}: Timing of intervention may matter more for glutamate/GABA patterns than dopamine patterns.

\section{Clinical Translation Framework}

\subsection{Why These Findings Do NOT Translate to Clinic (Yet)}

\subsubsection{Gene Enrichment $\neq$ Treatment Target}

\textbf{Common misconception}: ``High genetic signal $\rightarrow$ good drug target''

\textbf{Reality}: Genetic enrichment indicates:
\begin{itemize}
    \item Common variants near gene associated with disorder risk
    \item Effect sizes typically small (OR $\approx$ 1.05-1.2)
    \item Does NOT indicate:
    \begin{itemize}
        \item Direction of effect (gain vs. loss of function)
        \item Druggability of target
        \item Therapeutic window
        \item Off-target effects
    \end{itemize}
\end{itemize}

\subsubsection{No Patient Stratification}

\textbf{Critical limitation}: Gene patterns describe \textit{genes}, not \textit{patients}.

\textbf{Cannot answer}:
\begin{itemize}
    \item Which patients have glutamatergic vs. dopaminergic etiology?
    \item How to stratify for treatment selection?
    \item What biomarkers predict response?
\end{itemize}

\textbf{Would require}:
\begin{itemize}
    \item Individual-level genetic profiling
    \item Validation in independent cohorts
    \item Clinical trial data
    \item FDA approval process
\end{itemize}

\subsubsection{Direction of Effect Unknown}

\textbf{Problem}: GWAS identifies association, not mechanism.

\textbf{Example - GRIN2A}:
\begin{itemize}
    \item High genetic enrichment
    \item Could indicate:
    \begin{itemize}
        \item Too much NMDA receptor activity $\rightarrow$ need antagonist
        \item Too little NMDA receptor activity $\rightarrow$ need agonist
        \item Altered trafficking/localization $\rightarrow$ need modulator
    \end{itemize}
\end{itemize}

\textbf{Resolution}: Functional studies, expression data, animal models required.

\subsubsection{Clinical Trial Failures}

\textbf{Cautionary example - mGluR5}:
\begin{itemize}
    \item GRM5 shows highest enrichment (1144.2)
    \item Strong biological rationale (mGluR theory of autism)
    \item Clinical trial of mavoglurant (mGluR5 antagonist) $\rightarrow$ \textbf{FAILED}
    \item Negative results in fragile X syndrome and autism
\end{itemize}

\textbf{Lesson}: Genetic signal $\neq$ successful therapeutic intervention.

\subsection{What IS Scientifically Justified}

\subsubsection{Hypothesis Generation}

\textbf{Valid use}: Prioritize genes for basic research.

\textbf{Examples}:
\begin{itemize}
    \item Investigate GABRB2 (understudied relative to signal)
    \item Study GRIN2A mutations in patient-derived neurons
    \item Test GRM5 modulators in animal models
\end{itemize}

\subsubsection{Biological Insights}

\textbf{Valid use}: Understand shared vs. disorder-specific mechanisms.

\textbf{Examples}:
\begin{itemize}
    \item Glutamate/GABA → shared AuDHD biology
    \item Dopamine → ADHD-specific biology
    \item TPH2 → trans-diagnostic aggression/mood
\end{itemize}

\subsubsection{Research Stratification}

\textbf{Valid use}: Design studies testing pathway-specific hypotheses.

\textbf{Examples}:
\begin{itemize}
    \item MRI studies: glutamate spectroscopy in AuDHD vs. controls
    \item Animal models: GRIN2A knockouts + GABRB2 variants
    \item Drug repurposing: test GABAergic modulators in mouse models
\end{itemize}

\section{Evolutionary Perspective}

\subsection{Why Are Neurodevelopmental Risk Variants Common?}

\textbf{Paradox}: If ADHD/autism reduce fitness, why do risk variants persist at high frequency?

\textbf{Hypotheses}:

\textbf{1. Balancing selection}:
\begin{itemize}
    \item Heterozygotes have advantage
    \item Example: Glutamate receptor variants may balance excitability vs. stability
\end{itemize}

\textbf{2. Ancestral neutrality}:
\begin{itemize}
    \item Variants were neutral in ancestral environments
    \item Modern environments (education, social demands) reveal costs
\end{itemize}

\textbf{3. Pleiotropic benefits}:
\begin{itemize}
    \item Variants increase risk but also confer advantages
    \item Example: Dopamine variants may enhance novelty-seeking (adaptive in some contexts)
\end{itemize}

\textbf{4. Mutation-selection balance}:
\begin{itemize}
    \item Continuous generation of new variants
    \item Selection not strong enough to eliminate
\end{itemize}

\subsection{Cross-Species Conservation}

\textbf{Gene conservation scores}:
\begin{itemize}
    \item GRIN2A, GRM5, GABRB genes: Highly conserved across vertebrates
    \item Dopamine receptor genes: Less conserved; mammalian-specific isoforms
\end{itemize}

\textbf{Interpretation}:
\begin{itemize}
    \item Glutamate/GABA systems: Ancient, fundamental to nervous system
    \item Dopamine system: More recent elaboration; species-specific variations
\end{itemize}

\section{Multi-Omic Integration}

\subsection{What Additional Data Would Help}

\textbf{1. Gene expression}:
\begin{itemize}
    \item Brain region-specific expression (GTEx, Allen Brain Atlas)
    \item Developmental time courses (BrainSpan)
    \item Single-cell RNA-seq (cell type specificity)
\end{itemize}

\textbf{2. Protein-protein interactions}:
\begin{itemize}
    \item Physical interactions (BioGRID, STRING)
    \item Functional pathways (KEGG, Reactome)
    \item Post-synaptic density complexes
\end{itemize}

\textbf{3. Chromatin accessibility}:
\begin{itemize}
    \item ATAC-seq, ChIP-seq data
    \item Identify regulatory variants
    \item Tissue/cell-type specificity
\end{itemize}

\textbf{4. Clinical phenotypes}:
\begin{itemize}
    \item Symptom dimension scores
    \item Comorbidity patterns
    \item Treatment response data
\end{itemize}

\textbf{5. Imaging genetics}:
\begin{itemize}
    \item Structural MRI (cortical thickness, volumes)
    \item Functional MRI (connectivity, activation)
    \item MR spectroscopy (glutamate, GABA levels)
\end{itemize}

\subsection{Integration Approaches}

\textbf{Network-based methods}:
\begin{itemize}
    \item Build gene networks from multi-omic data
    \item Identify modules enriched for ADHD/autism risk
    \item Test if glutamatergic/GABAergic genes form coherent modules
\end{itemize}

\textbf{Machine learning}:
\begin{itemize}
    \item Train classifiers on genetic + clinical data
    \item Test if gene patterns predict:
    \begin{itemize}
        \item ADHD vs. autism vs. comorbid
        \item Treatment response
        \item Developmental trajectories
    \end{itemize}
\end{itemize}

\textbf{Mendelian randomization}:
\begin{itemize}
    \item Use genetic variants as instruments
    \item Test causal effects:
    \begin{itemize}
        \item Does GRIN2A expression causally affect ADHD risk?
        \item Mediation by intermediate phenotypes (e.g., cortical excitability)?
    \end{itemize}
\end{itemize}

\section{Ethical Considerations}

\subsection{Genetic Testing for ADHD/Autism}

\textbf{Current state}: NOT clinically useful.

\textbf{Why not}:
\begin{itemize}
    \item Polygenic architecture (thousands of variants, small effects)
    \item Predictive value low (area under ROC curve $<$ 0.6)
    \item No actionable information for treatment
\end{itemize}

\textbf{Future scenario}: If genetic stratification becomes validated:
\begin{itemize}
    \item \textbf{Benefits}: Personalized treatment selection
    \item \textbf{Risks}: Stigma, discrimination, deterministic thinking
\end{itemize}

\subsection{Neurodiversity vs. Medical Model}

\textbf{Tension}: Biological research vs. neurodiversity movement.

\textbf{Neurodiversity perspective}:
\begin{itemize}
    \item ADHD/autism are differences, not deficits
    \item Society should accommodate, not ``fix''
    \item Genetic research may pathologize natural variation
\end{itemize}

\textbf{Medical model perspective}:
\begin{itemize}
    \item ADHD/autism cause significant impairment
    \item Individuals deserve treatment options
    \item Understanding biology enables better interventions
\end{itemize}

\textbf{Integration}:
\begin{itemize}
    \item Research aims to reduce suffering, not eliminate diversity
    \item Treatments should be optional, not coercive
    \item Genetic insights can inform both medical and social interventions
\end{itemize}

\subsection{Responsible Communication}

\textbf{Avoid}:
\begin{itemize}
    \item Oversimplifying: ``ADHD is a dopamine disorder''
    \item Determinism: ``Your genes determine your fate''
    \item Premature clinical claims: ``Genetic test predicts treatment response''
\end{itemize}

\textbf{Emphasize}:
\begin{itemize}
    \item Complexity: Thousands of genes, environment matters
    \item Probabilistic: Genes influence risk, not destiny
    \item Limitations: Current findings are exploratory, require replication
\end{itemize}

\section{Limitations of Biological Interpretation}

\subsection{Inferential Gaps}

\textbf{Gap 1}: GWAS signal $\rightarrow$ Causal gene

\textbf{Problem}: LD structure means multiple genes in same region.

\textbf{Example}: GABRB3 locus includes multiple genes; attributing signal to GABRB3 alone may be incorrect.

\subsection{Ascertainment Bias}

\textbf{Gap 2}: Well-studied genes $\rightarrow$ More literature

\textbf{Problem}: Publication bias favors known genes.

\textbf{Example}: GABRB2 vs. GABRB3 disparity may partially reflect historical research focus, not just biological importance.

\subsection{Population Specificity}

\textbf{Gap 3}: European ancestry GWAS $\rightarrow$ Universal biology

\textbf{Problem}: Genetic architecture may differ across populations.

\textbf{Evidence}: Some GWAS hits fail to replicate in non-European cohorts.

\textbf{Implication}: Findings may not generalize globally.

\subsection{Simplification of Biology}

\textbf{Gap 4}: Gene list $\rightarrow$ Biological pathway

\textbf{Problem}: Pathways are interconnected; clean categories are artificial.

\textbf{Reality}: Genes interact in complex networks transcending simple pathway boundaries.

\section{Summary of Biological Insights}

\subsection{Core Findings}

\begin{enumerate}
    \item \textbf{Glutamatergic system}: Highest shared genetic enrichment; E/I imbalance hypothesis
    \item \textbf{GABAergic system}: High shared enrichment; inhibitory dysfunction
    \item \textbf{Serotonergic system}: Trans-diagnostic role (TPH2); aggression/mood
    \item \textbf{Dopaminergic system}: ADHD-specific; lower shared enrichment
    \item \textbf{Polygenetic background}: Mixed genes; likely general psychiatric risk
\end{enumerate}

\subsection{Novel Observations Requiring Follow-Up}

\begin{enumerate}
    \item \textbf{GABRB2 prominence}: Stronger signal than GABRB3 despite less prior literature
    \item \textbf{E/I imbalance}: Quantitative relationship between glutamate and GABA enrichments
    \item \textbf{Dopaminergic specificity}: Pattern suggests ADHD-specific vs. shared mechanisms
    \item \textbf{TPH2 consistency}: Trans-diagnostic effects across all disorder comparisons
\end{enumerate}

\subsection{What This Does NOT Tell Us}

\begin{itemize}
    \item Patient heterogeneity or subtypes
    \item Treatment response predictors
    \item Causal mechanisms (vs. correlations)
    \item Clinical actionability
    \item Interaction with environment
\end{itemize}

\subsection{Next Steps}

\begin{enumerate}
    \item \textbf{Replication}: Independent cohorts, diverse ancestries
    \item \textbf{Functional validation}: Cell models, animal studies
    \item \textbf{Clinical associations}: Genotype-phenotype correlations
    \item \textbf{Multi-omic integration}: Expression, protein, imaging data
    \item \textbf{Mechanistic studies}: How do variants affect protein function?
\end{enumerate}

\end{document}
